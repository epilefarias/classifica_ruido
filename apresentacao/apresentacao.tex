%%******************************************************************************
%% Instruções para compilação e apresentação
%%******************************************************************************

% para compilar use SHIFT + ALT + E (compilar direto para PDF)
%
% para apresentação use o adobe para evitar o flicker
%


\documentclass{beamer}


% ===============================================================
% compilar com XeLaTeX (Shift+Alt+E no WinEdt) em PORTUGUES
%\XeTeXinputencoding latin1
% ===============================================================



%%******************************************************************************
%% PACKAGES
%%******************************************************************************
\usepackage{beamerthemesplit}
\usepackage{beamerfoils}
\usepackage[T1]{fontenc}
\usepackage[latin1,utf8]{inputenc}
\usepackage[english,brazil]{babel}
\usepackage{indentfirst}         % indentacao de primeiro paragrafo
\usepackage{subfigure}
%\usepackage{subfig}
\usepackage{epsfig}
\usepackage{cite}
\usepackage{setspace}
\usepackage{rotating}
\usepackage{graphicx}
\usepackage{dsfont}
\usepackage{enumerate}
\usepackage{amsmath}
\usepackage{color}
\usepackage{bbm}
\usepackage{amssymb}
\usepackage{ulem}
\usepackage{comment}
\usepackage{glossaries}
\usepackage{pdflscape}
\usepackage{adjustbox}
\usepackage{tikz}
\usepackage{ragged2e}
\usepackage{gensymb}

\newcommand*\circled[1]{\tikz[baseline=(char.base)]{ \scriptsize
            \node[shape=circle,draw,inner sep=0.5pt] (char) {#1};}}



\mode<presentation>{

    %%******************************************************************************
    %% Temas e Cores
    %%******************************************************************************
    %%
    %% Temas
    %\usetheme{Berkeley}    % left bar
    %\usetheme{Antibes}     % tree header
    %\usetheme{Boadilla}     % BEST ----- very plain
    %\usetheme{Warsaw}
    %\usetheme{Rochester}
    %\usetheme{Madrid}
    %\usetheme{Goettingen}  % bad ---- right bar
    %\usetheme{Ilmenau}
    \usetheme{CambridgeUS} % good with dolphin
    %%
    %% Cores
    %\usecolortheme{whale}
    \usecolortheme{dolphin}
    %******************************************************************************




    %%******************************************************************************
    %% Setar a caixa de navegação
    %%******************************************************************************
    \usefonttheme[onlymath]{serif}
    %um dos dois abaixo
    \newcommand{\currentframe}{\hspace{2.3cm}\insertframenumber}%/\inserttotalframenumber}
    %\setbeamertemplate{footline}[page number]

    \setbeamertemplate{navigation symbols}{}
    \AtBeginSection[]
    {
    %\begin{frame}
    %\frametitle{Sumário}
    %\tableofcontents[currentsection,hideallsubsections]
    %\end{frame}
    }
    
    %\usepackage{beamerouterthemeshadow}
    %\usepackage{beamerouterthemesmoothtree}
    %\usepackage{beamercolorthemeorchid}
    %\usepackage{beamerinnerthemerounded}
    %\setbeamercovered{transparent}
    % comment out to completely hide covered material
}% end \mode<presentation>{


%*******************************************************************************
% não sei!!
%\usefonttheme[onlylarge]{structuresmallcapsserif}
\usepackage{times}
%*******************************************************************************




%%******************************************************************************
%% PÁGINA DO TÍTULO
%%******************************************************************************

\title[Ciência de Dados]{Comparação de Métodos de Classificação de Ruído Acústico}
%\title[Short Title]{Long Title}

\author[Nascimento, Farias, Alves]
{
Antônio Nascimento, Felipe Farias e Marília Alves
%Eddie B. L. Filho, Waldir S. S. Junior  \\
%\textit{Orientador}: Prof.Dr. Waldir Sabino da Silva Junior \\
%\textit{Co-Orientador}: Prof.Dr. Eddie Batista de Lima Filho
}


\institute[IME]{%
Instituto Militar de Engenharia \\
%Universidade do Estado do Amazonas
}

\date[Agosto de 2017] % (optional, should be abbreviation of conference name)
%{Simpósio Brasileiro de Telecomunicações \\ 
{Agosto de 2017}
% - Either use conference name or its abbreviation.


%%******************************************************************************
%% CORPO PRINCIPAL
%%******************************************************************************

\begin{document}

%\MyLogo{\includegraphics[scale=0.9]{figuras/logo-lab-apresentacao.eps}}

\justifying

\begin{frame}
  \titlepage
\end{frame}


%%==============================================================================
%% SEÇÃO Sumário
%%==============================================================================

%\MyLogo{\includegraphics[scale=0.5]{figuras/logo-lab-apresentacao.eps}}

\section[Sumário]{}
\begin{frame}
  \tableofcontents
\end{frame}



%%==============================================================================
%% SEÇÕES
%%==============================================================================

\section{Processamento de Sinais Acústicos em Ruído}


\begin{frame}
	\justifying
  	\frametitle{Tarefas de PDS}
  	
  	\begin{itemize}
  		 \setlength\itemsep{1em}
  		\item Tarefas de PDS  		
  	\end{itemize}
\end{frame}

\begin{frame}
	\justifying
  	\frametitle{Tarefas de PDS em ruído}
  	
  	\begin{itemize}
  		 \setlength\itemsep{1em}
  		\item Queda de desempenho nas tarefas de PDS 
  	\end{itemize}
\end{frame}

\begin{frame}
	\justifying
  	\frametitle{Classificação de Ruídos Acústicos}
  	
  	\begin{itemize}
  		 \setlength\itemsep{1em}
  		\item Objetivos do trabalho
 	\end{itemize}
  		
\end{frame}


\section{Extração de Atributos do Áudio}
  		
\begin{frame}
  	\frametitle{LPC}
  	\begin{itemize}
  		\setlength\itemsep{1em}
  		\item VAI QUE É TUA MARILIA
  		\item LPC
  	\end{itemize}
\end{frame}


\begin{frame}
  	\frametitle{MFCC}
  	\begin{itemize}
  		\setlength\itemsep{1em}
  		\item Coeficientes Mel-Cepstrais
  		\begin{itemize}
  			\item porque é melhor?
  		\end{itemize}
  	\end{itemize}
\end{frame}

\section{Experimentos}

\subsection{Base de Dados}

\begin{frame}
    \frametitle{A base de dados NOISEX-92}
    
    \begin{itemize}
    	\setlength\itemsep{1em}
        \item ela é linda
    	\item ela é cheirosa
	\end{itemize}
    
\end{frame}

\subsection{Validação Cruzada}

\begin{frame}
	\frametitle{Validação Cruzada}
	
	\begin{itemize}
		\item Explicar o \textit{K-fold} aqui
	\end{itemize}
\end{frame}

\subsection{Métodos Utilizados}

\begin{frame}

	\frametitle{K-Means 1 (intro) \textbf{EU NAO SEI O NOME DO ALGORITMO QUE A MARILIA USA}}
	
	\begin{itemize}	
		\item introdução bonitinha aqui
	\end{itemize}
	
\end{frame}


\begin{frame}

	\frametitle{K-Means 2 (especificações) \textbf{EU NAO SEI O NOME DO ALGORITMO QUE A MARILIA USA}}
	
	\begin{itemize}	
		\item detalhes das especificações aqui
	\end{itemize}
	
\end{frame}


\begin{frame}

	\frametitle{K-Means 3 (resultados) \textbf{EU NAO SEI O NOME DO ALGORITMO QUE A MARILIA USA}}
	
	\begin{itemize}	
		\item aquele tabelão maroto
	\end{itemize}
	
\end{frame}


\begin{frame}

	\frametitle{Gaussian Mixture Model 1 (intro)}
	
	\begin{itemize}	
		\item introdução bonitinha aqui
	\end{itemize}
	
\end{frame}


\begin{frame}

	\frametitle{Gaussian Mixture Model 2 (especificações)}
	
	\begin{itemize}	
		\item detalhes das especificações aqui
	\end{itemize}
	
\end{frame}


\begin{frame}

	\frametitle{Gaussian Mixture Model 3 (resultados)}
	
	\begin{itemize}	
		\item aquele tabelão maroto
	\end{itemize}
	
\end{frame}


\begin{frame}

	\frametitle{Neural Network 1 (intro)}
	
	\begin{itemize}	
		\item introdução bonitinha aqui
	\end{itemize}
	
\end{frame}


\begin{frame}

	\frametitle{Neural Network 2 (especificações)}
	
	\begin{itemize}	
		\item detalhes das especificações aqui
	\end{itemize}
	
\end{frame}


\begin{frame}

	\frametitle{Neural Network 3 (resultados)}
	
	\begin{itemize}	
		\item aquele tabelão maroto
	\end{itemize}
	
\end{frame}


\begin{frame}

	\frametitle{Support Vector Machines 1 (intro)}
	
	\begin{itemize}	
		\item introdução bonitinha aqui
	\end{itemize}
	
\end{frame}


\begin{frame}

	\frametitle{Support Vector Machines 2 (especificações)}
	
	\begin{itemize}	
		\item detalhes das especificações aqui
	\end{itemize}
	
\end{frame}


\begin{frame}

	\frametitle{Support Vector Machines 3 (resultados)}
	
	\begin{itemize}	
		\item aquele tabelão maroto
	\end{itemize}
	
\end{frame}

\begin{frame}

	\frametitle{Comparação}
	
	\begin{itemize}	
		\item Tabela com a comparação entre os modelos.
	\end{itemize}
	
\end{frame}

\section{Considerações Finais}

\begin{frame}

	\frametitle{Conclusões}
	
	\begin{itemize}
		\setlength\itemsep{1em}
		
		\item Concluímos que o trabalho foi muito trabalhoso.
	
	\end{itemize}


\end{frame}

\begin{frame}

	\frametitle{Trabalhos Futuros}
	
	\begin{itemize}
		\setlength\itemsep{1em}
		
		\item Extender a comparação a outros métodos de classificação.
		\item Investigar desempenho de comitê de classificação.
	
	\end{itemize}


\end{frame}


\begin{frame}
	\frametitle{Obrigado!}
    
    \centering
    
    \begin{itemize}
    	\setlength\itemsep{2em}
		\item Valeu galera!
	\end{itemize}

\end{frame}

\end{document}
